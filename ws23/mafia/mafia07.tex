\documentclass{article}
\usepackage[utf8]{inputenc}
\usepackage{amsmath, amssymb, amsthm}
\usepackage{geometry}
\usepackage{microtype}
\usepackage{booktabs}
\usepackage[ngerman]{babel}
\usepackage{setspace}

% Anpassen der Seitenränder
\geometry{left=25mm, right=25mm, top=25mm, bottom=25mm}

% Einstellen des Zeilenabstandes
\setstretch{1.25}

\title{Lösungen zu Übungsaufgaben 07 \\ \small Gruppe: Mi 08-10 SR 2, Barbara Rieß}
\author{Linus Keiser}
\date{13. Dezember 2023}

% Theorem-Umgebungen
\renewcommand{\proofname}{Beweis}
\newtheorem{theorem}{Satz}
\theoremstyle{definition}
\newtheorem{definition}{Definition}
\theoremstyle{remark}
\newtheorem*{remark}{Bemerkung}

\begin{document}

\maketitle
\section*{Aufgabe 25}

\subsection*{(a) Betragssummennorm}

\textit{Zu zeigen:} die Abbildung \( \| x \|_1 := \sum_{k=1}^{n} |x_k| \) für \( x = (x_1, \ldots, x_n)^T \in \mathbb{R}^n \) eine Norm auf \( \mathbb{R}^n \) definiert.

\begin{proof} Wir überprüfen die Normeigenschaften (N1) bis (N4).

	\textbf{(N1) Positivität:}
	Da der Betrag einer jeden reellen Zahl nichtnegativ ist, folgt, dass die Summe der Beträge der Komponenten von \( x \) ebenfalls nichtnegativ ist. Daher gilt \( \| x \|_1 \geq 0 \) für alle \( x \in \mathbb{R}^n \).

	\textbf{(N2) Definitheit:}
	Es gilt \( \| x \|_1 = 0 \) genau dann, wenn jeder Betrag \( |x_k| = 0 \) für \( k = 1, \ldots, n \) ist. Dies ist genau dann der Fall, wenn jedes \( x_k = 0 \) ist. Daher ist \( \| x \|_1 = 0 \) genau dann, wenn \( x = 0 \).

	\textbf{(N3) Homogenität:}
	Für ein beliebiges \( \alpha \in \mathbb{R} \), betrachten wir \( \| \alpha x \|_1 = \sum_{k=1}^{n} |\alpha x_k| \). Aufgrund der Eigenschaften des Betrags gilt \( |\alpha x_k| = |\alpha||x_k| \). Daher ist \( \| \alpha x \|_1 = |\alpha| \sum_{k=1}^{n} |x_k| = |\alpha| \| x \|_1 \).

	\textbf{(N4) Dreiecksungleichung:}
	Für \( x, y \in \mathbb{R}^n \), gilt \( \| x + y \|_1 = \sum_{k=1}^{n} |x_k + y_k| \). Aufgrund der Dreiecksungleichung für Beträge folgt \( |x_k + y_k| \leq |x_k| + |y_k| \). Daher ist \( \| x + y \|_1 \leq \sum_{k=1}^{n} (|x_k| + |y_k|) = \| x \|_1 + \| y \|_1 \).

	Damit sind die Normeigenschaften für \( \| x \|_1 \) gezeigt.
\end{proof}

\subsection*{(b) Maximumsnorm}

\textit{Zu zeigen:} die Abbildung \( \| x \|_{\infty} := \max \{ |x_1|, |x_2|, \ldots, |x_n| \} \) für \( x = (x_1, \ldots, x_n)^T \in \mathbb{R}^n \) eine Norm auf \( \mathbb{R}^n \) definiert.

\begin{proof} Hierbei überprüfen wir erneut die Normeigenschaften (N1) bis (N4).

	\textbf{(N1) Positivität:}
	Das Maximum einer Menge nichtnegativer Zahlen ist nichtnegativ. Da die Beträge der Komponenten von \( x \) nichtnegativ sind, folgt, dass \( \| x \|_{\infty} \geq 0 \) für alle \( x \in \mathbb{R}^n \).

	\textbf{(N2) Definitheit:}
	Wenn \( \| x \|_{\infty} = 0 \), dann ist das Maximum der Beträge der Komponenten von \( x \) null. Dies bedeutet, dass jede Komponente \( x_k \) null sein muss, und somit ist \( x = 0 \).

	\textbf{(N3) Homogenität:}
	Für ein beliebiges \( \alpha \in \mathbb{R} \), betrachten wir \( \| \alpha x \|_{\infty} \). Es gilt \( \| \alpha x \|_{\infty} = \max \{ |\alpha x_1|, ..., |\alpha x_n| \} = |\alpha| \max \{ |x_1|, ..., |x_n| \} = |\alpha| \| x \|_{\infty} \), was aus den Eigenschaften des Betrags folgt.

	\textbf{(N4) Dreiecksungleichung:}
	Für \( x, y \in \mathbb{R}^n \) gilt \( \| x + y \|_{\infty} = \max \{ |x_1 + y_1|, ..., |x_n + y_n| \} \). Unter Anwendung der Dreiecksungleichung für Beträge ergibt sich \( |x_k + y_k| \leq |x_k| + |y_k| \). Daher ist \( \| x + y \|_{\infty} \leq \max \{ |x_k| + |y_k|, ..., |x_n| + |y_n| \} \). Da für jede Komponente gilt, dass \( |x_k|, |y_k| \leq \| x \|_{\infty}, \| y \|_{\infty} \), folgt, dass \( \| x + y \|_{\infty} \leq \| x \|_{\infty} + \| y \|_{\infty} \).

	Damit sind die Normeigenschaften (N1) bis (N4) für \( \| x \|_{\infty} \) gezeigt.
\end{proof}


\section*{Aufgabe 26}

\textit{Ziel:} Umformung in die Form \( z = x + iy \) und Berechnung von Realteil, Imaginärteil und Betrag.

\subsubsection*{(a) \( z = (3 + 4i)(2 - i)^{2} - (5 - i) + 27\)}

Zunächst erweitern wir den quadratischen Term \((2 - i)^{2}\). Unter Verwendung der binomischen Formel und der Eigenschaft \(i^2 = -1\) gilt:
\begin{align*}
	(2 - i)^2 & = (2 - i)(2 - i)                \\
	          & = 2^2 - 2 \cdot 2 \cdot i + i^2 \\
	          & = 4 - 4i - 1                    \\
	          & = 3 - 4i.
\end{align*}
Durch Multiplikation mit dem verbundenen Term \( (3 + 4i) \) erhalten wir folglich:
\begin{align*}
	(3 + 4i)(3 - 4i) & = (3 + 4i) \cdot (3 - 4i) \\
	                 & = 25.
\end{align*}
Zusammen mit den anderen Termen ergibt sich:
\begin{align*}
	z & = 25 - (5 - i) + 27 \\
	  & = 25 - 5 + i + 27   \\
	  & = 47 + i.
\end{align*}
Daher ist der Realteil \( x = 47 \), der Imaginärteil \( y = 1 \) und der Betrag von \( z \) ist \( |z| = \sqrt{47^2 + 1^2} = \sqrt{2210} \).

\subsubsection*{(b) \( z = \frac{7 - 3i}{6i - 4} \)}

Wir vereinfachen den Bruch durch Multiplikation von Zähler und Nenner mit der konjugierten Zahl des Nenners. Dies führt zu einem reellen Nenner. Die konjugierte Zahl zu \( 6i - 4 \) ist \( -4 + 6i \). Damit ist der vereinfachte Bruch
\begin{align*}
	z & = \frac{7 - 3i}{6i - 4} \cdot \frac{-6i - 4}{-6i - 4} \\
	  & = \frac{(7 - 3i)(-6i - 4)}{(6i - 4)(-6i - 4)}.
\end{align*}
Durch Multiplikation im Zähler und Nenner erhalten wir:
\begin{align*}
	\text{Zähler: } & (7 - 3i)(-6i - 4) = 7(-6i) - 7 \cdot 4 - 3i(-6i) - 3i \cdot 4 \\
	                & = -42i - 28 + 18i^2 - 12i                                     \\
	                & = -42i - 28 - 18 - 12i                                        \\
	                & = -46 - 54i.                                                  \\
	\text{Nenner: } & (6i - 4)(-6i - 4) = 6i(-6i) - 6i \cdot 4 - 4(-6i) - 4 \cdot 4 \\
	                & = -36i^2 + 24i + 24i + 16                                     \\
	                & = 36 + 48i + 16                                               \\
	                & = 52.
\end{align*}
Vereinfacht ergibt sich:
\begin{align*}
	z & = \frac{-46 - 54i}{52}             \\
	  & = -\frac{23}{26} - \frac{15i}{26}.
\end{align*}
Damit haben wir also:
\begin{align*}
	\text{Realteil: }     & x = -\frac{23}{26},                                                                                                    \\
	\text{Imaginärteil: } & y = -\frac{15}{26},                                                                                                    \\
	\text{Betrag: }       & |z| = \sqrt{x^2 + y^2} = \sqrt{\left(-\frac{23}{26}\right)^2 + \left(-\frac{15}{26}\right)^2} = \frac{\sqrt{754}}{26}.
\end{align*}

\section*{Aufgabe 27}

\subsection*{(a)}
Wir zeigen, dass für alle \( z, w \in \mathbb{C} \) die Regel \( \overline{z + w} = \overline{z} + \overline{w} \) gilt.

\proof Es sei
\begin{enumerate}
	\item	\( z = x + yi \), wobei \( x \) der Realteil und \( y \) der Imaginärteil von \( z \) ist (und \( i \) die imaginäre Einheit).
	\item	\( w = u + vi \), wobei \( u \) der Realteil und \( v \) der Imaginärteil von \( w \) ist.
\end{enumerate}
Wir berechnen die linke und rechte Seite der Gleichung \( \overline{z + w} = \overline{z} + \overline{w} \) und zeigen, dass beide Seiten identisch sind.
Für die linke Seite gilt:
\[ z + w = (x + yi) + (u + vi) = (x + u) + (y + v)i. \]
Dann ist \( \overline{z + w} = \overline{(x + u) + (y + v)i} = (x + u) - (y + v)i \). Für die rechte Seite \( \overline{z} + \overline{w} \) gilt:
\begin{align*}
	\overline{z} + \overline{w} & = \overline{x + yi} + \overline{u + vi} \\
	                            & = x - yi + u - vi                       \\
	                            & = (x + u) - (y + v)i.
\end{align*}
Wir sehen, dass die linke und rechte Seite identisch sind. Daher ist die Gleichung \( \overline{z + w} = \overline{z} + \overline{w} \) für alle \( z, w \in \mathbb{C} \) wahr.
\endproof

\subsection*{(b)}
Wir zeigen, dass für alle \( z, w \in \mathbb{C} \) die Regel \( \overline{z \cdot w} = \overline{z} \cdot \overline{w} \) gilt.

\proof Es gelte die Definition von \( z \) und \( w \) wie in (a). Wir berechnen die linke und rechte Seite der Gleichung \( \overline{z \cdot w} = \overline{z} \cdot \overline{w} \) und zeigen, dass beide Seiten identisch sind.
Zuerst berechnen wir wir:
\begin{align*}
	z \cdot w & = (x + yi) \cdot (u + vi)                                        \\
	          & = xu + xvi + yiu + yvi^2                                         \\
	          & = (xu - yv) + (xv + yu)i. \quad \text{Da } i^2 = -1 \text{ ist.}
\end{align*}
Dann ist
\begin{align*}
	\overline{z \cdot w} & = \overline{(xu - yv) + (xv + yu)i} \\
	                     & = (xu - yv) - (xv + yu)i.
\end{align*}
Für die rechte Seite \( \overline{z} \cdot \overline{w} \) gilt:
\begin{align*}
	\overline{z} \cdot \overline{w} & = \overline{x + yi} \cdot \overline{u + vi} \\
	                                & = x - yi \cdot u - vi                       \\
	                                & = xu - xvi - yui + yvi^2                    \\
	                                & = (xu - yv) - (xv + yu)i.
\end{align*}
Wir sehen, dass die linke und rechte Seite identisch sind. Daher ist die Gleichung \( \overline{z \cdot w} = \overline{z} \cdot \overline{w} \) für alle \( z, w \in \mathbb{C} \) wahr.
\endproof

\subsection*{(c)}
Wir zeigen, dass für alle \( z \in \mathbb{C} \) die Regel \( z + \overline{z} = 2 \operatorname{Re}(z) \) gilt.

\proof Es gelte die Definition von \( z \) wie in (a).
Das komplex Konjugierte von \( z \) ist \( \overline{z} = x - yi \). Dann ist
\begin{align*}
	z + \overline{z} & = (x + yi) + (x - yi) \\
	                 & = 2x.
\end{align*}
Der Realteil von \( z \) ist \( x \), also ist \( 2x = 2 \operatorname{Re}(z) \). Daher ist die Gleichung \( z + \overline{z} = 2 \operatorname{Re}(z) \) für alle \( z \in \mathbb{C} \) wahr.
\endproof

\subsection*{(d)}
Wir zeigen, dass für alle \( z \in \mathbb{C} \) die Regel \( z - \overline{z} = 2i \operatorname{Im}(z) \) gilt.

\proof Es gelte die Definition von \( z \) wie in (a).
Das komplex Konjugierte von \( z \) ist \( \overline{z} = x - yi \). Dann ist
\begin{align*}
	z - \overline{z} & = (x + yi) - (x - yi) \\
	                 & = 2yi.
\end{align*}
Der Imaginärteil von \( z \) ist \( y \), also ist \( 2yi = 2i \operatorname{Im}(z) \). Daher ist die Gleichung \( z - \overline{z} = 2i \operatorname{Im}(z) \) für alle \( z \in \mathbb{C} \) wahr.
\endproof

\subsection*{(e)}
Wir zeigen, dass für alle \( z \in \mathbb{C} \) die Regel \( z \overline{z} \ge 0 \) und \( z \overline{z} = 0 \leftrightarrow z = 0\) gilt.

\proof Es gelte wieder die Definition von \( z \) wie in (a).
Für \( z \overline{z} \) gilt:
\begin{align*}
	z \overline{z} & = (x + yi)(x - yi) \\
	               & = x^2 + y^2.
\end{align*}
Wir wissen, dass \( z \overline{z} \ge 0\), weil sowohl \( x^2 \) als auch \( y^2 \) als Quadrate reeler Zahlen immer positiv sind, also ist auch die Summe \( x^2 + y^2 \ge 0 \).
Für die Bedingung der Äquivalenz gilt für die Hinrichtung, dass wenn \( z \overline{z} = 0 \) ist, dann muss \( x^2 + y^2 = 0\) sein. Da Quadrate nur dann null sind, wenn die Basis null ist, folgt \( x = 0 \) und \( y = 0 \). Also ist \( z = 0 \).
Für die Rückrichtung gilt, dass wenn \( z = 0 \) ist, dann ist offensichtlich \( x = 0 \) und \( y = 0 \), und somit ist\( z \overline{z} = x^2 + y^2 = 0 \).
Damit haben wir gezeigt, dass \( z \overline{z} \) immer nichtnegativ ist und nur dann null wird, wenn \( z \) selbst null ist.
\endproof

\section*{Aufgabe 28}

\subsection*{(a)}

\begin{theorem*}
	Es gilt \( z^n = |z|^n(\cos(n \varphi) + i \sin(n \varphi))\) für alle \( z = |z|(\cos(\varphi) + i \sin(\varphi)) \in \mathbb{C} \) und \( n \in \mathbb{N} \).
\end{theorem*}

\begin{proof} Wir bestimmen
	
\end{proof}
\end{document}




